%%% START B [ 678 words ] 

In robot motion planning, the degrees of freedom determined by a robot's effectors define a vector space called the {\it{configuration space}} of the physical system and the reachable space accounting for nearby obstacles and mechanical limitations is a lower-dimensional manifold called the {\it{configuration manifold}}. Any possible movement the robot can make corresponds to path on the configuration manifold.

The above characterization of robot motion planning doesn't account for the role of perception and feedback. Robust motion plans must necessarily include perceptual activities whose purpose it is to help guide movement, avoid obstacles including one's own body parts and facilitate accurate positioning and grasping.      

To serve as a basis for motion planning, an internal representation must capture not only the physical constraints that the body and environment impose on movement, but also the cognitive state of the agent incorporating current estimates of the location and velocity of nearby objects and body parts relative to the body's frame of reference and relevant knowledge about their properties and role in the agent's objectives.

Motor plans are represented in the primary, premotor and supplementary motor cortex in the frontal cortex adjacent to the central sulcus. The nearby somatosensory and proprioceptive association areas of the anterior parietal cortex are likely to figure prominently in planning movement, as are the sensory association areas in the posterior parietal cortex and in particular the dorsal visual stream also known as the "where" or "how" stream. These sources of relevant neural activity provide the basal ganglia with a rich context for guiding action selection.

We know that regions of the motor cortex are topographically organized, effectively combining information from sensorimotor areas throughout cortex to create a composite map or {\it{augmented configuration manifold}} for planning purposes. In lieu of an internal representation for abstract goals, we borrow the term {\it{setpoint}} from control theory to denote the target state provided by basal ganglia. Negative feedback systems use the difference between the system state and the setpoint to guide action selection and are common in biological organisms~\cite{Ashby1957cybernetics}.

The complete context for acting is a composite summary of the agent's sensorimotor, proprioceptive and vestibular state derived from a hierarchy of primary, secondary and associative features that constitute an abstraction hierarchy~\cite{FusterPREFRONTAL-CORTEX-15-CHAPTER_8} and roughly aligns with the features available to the basal ganglia and prefrontal cortex for modulating action selection.

The division of labor between the basal ganglia, motor cortex and cerebellar cortex is pretty well established. The basal ganglia do not directly select motor programs but rather they enable them to run in the motor cortex. The motor cortex selects and executes motor programs issuing motor commands via the descending pathways. The cerebellum does not initiate motor commands but rather modifies the motor commands of the descending pathways to make movements more adaptive and accurate. 

In the model for motor control presented here, the basal ganglia initiate a motor program in the motor cortex by creating a context for action corresponding to a desired future state. Since the motor cortex already has access to this information, all that is required of the basal ganglia is an offset from the current context to serve as a setpoint, thereby enabling the motor cortex to select an appropriate motor program for generating motor commands. Execution then consists of repeatedly invoking the selected motor program to traverse the augmented configuration manifold from the current context to the specified offset / setpoint.

The actions available in the process of executing a motor program include motor commands in the form of muscle contraction and relaxation and sensorimotor activities in service to visual {\emdash{}} or other sensory modality {\emdash{}} servoing of the sort required for grasping objects and avoiding obstacles. Methods for path planning such as artificial potential field methods~\cite{KhatibIJRR-86} or strategies for sensor-based traversals with randomized recovery could easily be incorporated in this model~\cite{LiarokapisetalICAR-15}. As for learning how to carry out and coordinate more complicated movements and manipulations, decades of research developing models of the cerebellum offer practical suggestions.

%%% STOP B 